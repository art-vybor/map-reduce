% utf-8 ru, unix eolns
\documentclass[12pt,a4paper,oneside]{extarticle}
    \righthyphenmin=2 %минимально переносится 2 символа %%%
    \sloppy

\title{Курсач: <<Молекулярное моделирование>>}

% Рукопись оформлена в соответствии с правилами оформления 
% электронной версии авторского оригинала, 
% принятыми в Издательстве МГТУ им. Н.Э. Баумана.

\usepackage{geometry} % А4, примерно 28-31 строк(а) на странице 
    \geometry{paper=a4paper}
    \geometry{includehead=false} % Нет верх. колонтитула
    \geometry{includefoot=true}  % Есть номер страницы
    \geometry{bindingoffset=0mm} % Переплет    : 0  мм
    \geometry{top=20mm}          % Поле верхнее: 20 мм
    \geometry{bottom=25mm}       % Поле нижнее : 25 мм 
    \geometry{left=25mm}         % Поле левое  : 25 мм
    \geometry{right=25mm}        % Поле правое : 25 мм
    \geometry{headsep=10mm}  % От края до верх. колонтитула: 10 мм
    \geometry{footskip=20mm} % От края до нижн. колонтитула: 20 мм 

\usepackage{cmap}
\usepackage[T2A]{fontenc} 
\usepackage[utf8x]{inputenc}
\usepackage[english,russian]{babel}
\usepackage{misccorr}

\usepackage{amsmath}
\usepackage{amsfonts}
\usepackage{amssymb}

%\usepackage{cm-super} %человеческий рендер русских шрифтов

\setlength{\parindent}{1.25cm}  % Абзацный отступ: 1,25 см
\usepackage{indentfirst}        % 1-й абзац имеет отступ

\usepackage{setspace}   

\onehalfspacing % Полуторный интервал между строками

\makeatletter
\renewcommand{\@oddfoot }{\hfil\thepage\hfil} % Номер стр.
\renewcommand{\@evenfoot}{\hfil\thepage\hfil} % Номер стр.
\renewcommand{\@oddhead }{} % Нет верх. колонтитула
\renewcommand{\@evenhead}{} % Нет верх. колонтитула
\makeatother

\usepackage{fancyvrb}

\usepackage[pdftex]{graphicx}  % поддержка картинок для пдф
%\graphicspath{{noiseimages/}} %картинки в папке с tex файлом
%\DeclareGraphicsExtensions{.jpg,.png}


\renewcommand{\labelenumi}{\theenumi.} %меняет вид нумерованного списка

\usepackage{perpage} %нумерация сносок 
\MakePerPage{footnote}

\usepackage[all]{xy} %поддержка графов

\usepackage{listings} %листинги


\usepackage{url}


\usepackage{tikz} %для рисования графиков
\usepackage{pgfplots}


\usepackage{ccaption}%изменяет подпись к рисунку
\makeatletter 
\renewcommand{\fnum@figure}[1]{Рисунок~\thefigure~---~\sffamily}
\makeatother

\begin{document}
\pgfplotsset{compat=1.8}

\thispagestyle{empty}
\newpage
{
\centering


\textbf{
МОСКОВСКИЙ ГОСУДАРСТВЕННЫЙ ТЕХНИЧЕСКИЙ УНИВЕРСИТЕТ ИМЕНИ Н. Э. БАУМАНА \\
Факультет информатики и систем управления \\
Кафедра теоретической информатики и компьютерных технологий}
\bigskip
\bigskip
\bigskip
\bigskip
\bigskip
\bigskip
\bigskip

\vfill


Курсовой проект \\
по курсу <<Компьютерные системы и сети>>

\bigskip

{\large <<Фреймворк и файловая система для распределённой обработки больших данных в рамках концепции map-reduce>>}
\bigskip

\vfill



\hfill\parbox{4cm} {
Выполнил:\\
студент ИУ9-91 \hfill \\
Выборнов А. И.\hfill \medskip\\
Руководитель:\\
Дубанов А. В.\hfill
}


\vspace{\fill}

Москва \number\year
\clearpage
}


\tableofcontents

\clearpage


\section*{Введение}
\addcontentsline{toc}{section}{Введение}
   
\clearpage

\section{Теоретическая часть}
    \subsection{Map-reduce}
        \begin{itemize}
            \item Структура (key, value) - пара (ключ, значение).
            \item Программирование представляет собой определение двух функций:
            \begin{itemize}
                \item $map: (key, value)\rightarrow[(key, value)]$
                \item $reduce: (key, [value])\rightarrow[(key, value)]$
            \end{itemize}
            \item Между стадиями $map$ и $reduce$ происходит группировка и сортировка данных.
        \end{itemize}

        Картинка иллюстрирующая процесс, более подробное описание как он работает.

        \subsubsection{Зачем нужен map-reduce}
            \begin{itemize}
                \item Обработка больших данных (Big Data).
                \begin{itemize}
                    \item Вычисления превосходят возможности одной машины.  
                    \item Данные не помещаются в памяти, необходимо обращаться к диску.
                    \item Можно хранить много данных, но задержки и пропускная способность оборудования растут пропорционально данным.
                \end{itemize}
                \item Удобная абстракция для построения алгоритмов обработки больших данных.
                \item Устойчивость к отказам.            
            \end{itemize} 

        \subsubsection{Пример применения map-reduce}
            \begin{itemize}
                \item {\bfЗадача:} Есть граф пользователей некоторого ресурса, заданный в виде строчек: <<пользователь - друг1 друг2 ...>>. Для каждой пары пользователей найти общих друзей.
                \item Разберём задачу на следующих входных данных: \\
                    \begin{itemize}
                        \item A - B C D
                        \item B - A C
                        \item C - A B D
                        \item D - A C
                    \end{itemize}          
                \item На стадии map преобразовываем пару (пользователь, друзья) в множество пар следующим образом:
                \begin{itemize}
                    \item (A, B C D) -> (A B, B C D), (A C, B C D), (A D, B C D)
                    \item (B, A C) -> (A B, A C), (B C, A C)
                    \item (C, A B D) -> (A C, A B D), (B C, A B D), (C D, A B D)
                    \item (D, A C) -> (A D, A C), (C D, A C)
                \end{itemize}
                \item Сливаем результаты полученные на стадии map, получаем список пар:
                \begin{itemize}
                    \item (A B, [B C D, A C])
                    \item (A C, [B C D, A B D])
                    \item (A D, [B C D, A C])
                    \item (B C, [A B D, A C])
                    \item (C D, [A B D, A C])
                \end{itemize}
                \item На стадии reduce пересекаем с друг другом все элементы списка значений и получаем:
                \begin{itemize}
                    \item (A B, C)
                    \item (A C, B D)
                    \item (A D, C)
                    \item (B C, A)
                    \item (C D, A)
                \end{itemize}
            \end{itemize}
    \clearpage

    \subsection{Распределённая файловая система}
        Что такое, зачем требуется для данного проекта
        \subsubsection{Архитектура распределённой файловой системы}
            картинка с описанием
    \clearpage
    \subsection{Распределённый map-reduce}
        Что такое, зачем он нужен
        map-reduce удобная концепция, но ...
        
        \subsubsection{Архитектура распределённого map-reduce}
            картинка с подробным описанием
    \clearpage

    \subsection{Решаемый класс задач}
        мат выкладки
    \clearpage
\clearpage

\section{Объекты и методы}      
        \noindent Характеристики программного обеспечения:
        \begin{itemize}
            \item Операционная система --- Ubuntu 14.04.1 LTS 64-bit.
            \item IDE --- Syblime Text 2.
            \item Язык программирования --- Python 2.7.3.
        \end{itemize}
        
        \noindent Характеристики оборудования:
        \begin{itemize}
            \item Процессор --- Intel Core i7-3770k 3.5Ghz$\times$8.
            \item Оперативная память --- 16Gb DDR3.
            \item Видеокарта --- ATI Radeon 7860.
        \end{itemize}
\clearpage

\section{Реализация}
    \subsection{Используемые технологии}
        внешние технологии используемые в проекте
        сериализация, zmq и прочее
        Парам пам пам
    \clearpage

    \subsection{Работа с большими данными}
        какие есть проблемы
        \subsubsection{python generator}
            nee
        \subsubsection{split}
            бла

        \subsubsection{dfs}
            проблемы в dfs
            как эти проблемы решаются в фс
        \subsubsection{map-reduce}
            проблемы в map-reduce
            как эти проблемы решаются в map-reduce
    \clearpage 

    \subsection{Взаимодействие между узлами}
        Описание реализации взаимодействия между различными узлами сети.
        \subsubsection{класс nodesmanager}
            описание класса
    \clearpage

    \subsection{Интерфейс}
        есть dfs, есть mr
        \subsubsection{dfs}


            Distributed file system is required to map-reduce framework.

            On each node, you should run *dfsnode.py* with two arguments - port and storagepath. Like this:


                python dfsnode.py -p 5556 -s /home/username/storage


            Then you should fill *config.json* with information about nodes. Now you can use *dfs.py*. Samples of use dfs.py:


                python dfs.py -ls /user/
                python dfs.py -mkdir /user/username/userdatafolder
                python dfs.py -put ./test.in /user/username/userdatafolder/testfile
                python dfs.py -get /user/username/userdatafolder/testfile
                python dfs.py -rm /user/username
        \subsubsection{mr}
    \clearpage  


\clearpage

\section{Тестирование}
    
        
\clearpage

\section{Заключение}
    
\clearpage


\begin{thebibliography}{0}
\addcontentsline{toc}{section}{Список литературы}
    \bibitem{SMILES}
         SMILES~---~A Simplified Chemical Language~//~Daylight Chemical Information Systems, Inc: URL: http://www.daylight.com/dayhtml/doc/theory/theory.SMILES.html
    \bibitem{PDBformat}
        Atomic Coordinate Entry Format Description~//~Penn State University: URL: http://www.wwpdb.org/documentation/format33/v3.3.html
        
    \bibitem{PSU}
        Periodic Table Datan Files~//~Protein Data Bank: URL: http://php.scripts.psu.edu/djh300/cmpsc221/p3s11-pt-data.htm
    \bibitem{threejs}
        Three.js~---~javascript 3D library~//~Three.js: URL: http://mrdoob.github.io/three.js/
        
    \bibitem{wikiribbon}
       File: Tubby-1c8z-pymol.png~//~Wikipedia: URL: http://en.wikipedia.org/wiki/File:Tubby-1c8z-pymol.png
             
    \bibitem{GLMol}
       GLmol - Molecular Viewer on WebGL/Javascript~//~GLmol: URL: http://webglmol.sourceforge.jp/index-en.html
        
\end{thebibliography}

\end{document}




